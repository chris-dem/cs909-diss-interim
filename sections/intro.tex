Over the last couple of years, there has been a revolutionary
initiative in the field of combinatronics.
Combinatronics has been a field of study in mathematics that primarily focused on
the notion of counting objects with certain properties. Over time, this notion
has shifted, especially in the subfield of algebraic combinatronics, where
there is no clear notion of the object that we are counting,
and the numbers express something more abstract \cite{pakWhatCombinatorialInterpretation2022}. %% Add the other two missing papers: Igors and the other guy
This gave a need to be able to assign a combinatorial interpretation to such numbers, or
more simply, do these numbers correspond to some counting over a set of objects.
Being able to find such definitions or interpretations can be very important,
it allows us to utilise tools from combinatronics as well as allow us to understand
and reveal hidden structures and properties for such numbers \cite{pakWhatCombinatorialInterpretation2022}. 
Moreover, there are several problems or numbers such as \textit{Kronecker coefficients},
whose combinatorial interpretation, would give a step towards the resolution of the $\textit{P} \neq \textit{NP}$
conjecture \cite{ikenmeyerWhatWhatNot2022a}.


To reiterate the previous statement, we can understand combinatorial interpretation as
the process of: given a sequence of numbers $\{a_x\}$,
find a set of combinatorial objects $A_x$ such that $|A_x| = a_x$
To formalise the current idea, Igor Pak et al. has concluded that
$f \in \textit{\#P}$, implies that $f$ has a combinatorial interpretation
\cite{pakWhatCombinatorialInterpretation2022, ikenmeyerWhatWhatNot2022a}.
We will explore this idea in much greater detail in the upcoming section,
but the main benefit is the ability to use a very expressive but formal language
that encapsulates this notion of a combinatorial interpretation.


In our current work, we focus on extending the work done by
Ikenmeyer et al., where they focused on the creation of frameworks
that determine whether $f \in^? \textbf{\#P}$, by looking
at the complexity class of $\textbf{\#}\textit{TFNP -1}$.
This is a class of problems that are guaranteed to have a solution
and their solutions are verifiable in polynomial time.
In their paper, they were able to show that for a subclass of
problems, also known as $\textit{PPAD} \subseteq \textit{TFNP}$,
different $\textit{PPAD-complete}$ problems, may or may not have a combinatorial
interpretation. Our contribution, comes to the analysis of a specific
problem, known as $\text{PureCircuit}$, which utilises
Kleene logic, to find satisfying assingments in sequential circuits.
We hope to demonstrate that such problem could help us bound, the
counting complexity limits of $\textbf{\#}\text{PureCircuit} -1$.

\subsection{Project objectives}

Below we will present our table of objectives. We will denote updated objectives with
(*), new objectives with (!), completed objectives with (\checkmark),  deleted objectives with (-).
%
% \begin{enumerate}[label*=R.\arabic*)]
%     \item (\checkmark) Find a parsimonious reduction from the $\textit{EndOfLine}$ to $\textit{EndOfLine}$.
%     \item (!) Improve the combinatorial bound between the reduction from EndOfLine to PureCircuit
%     \item (!) Demonstrate that $\textbf{\#}\text{PPAD}(PureCircuit)- 1 \not\subseteq \textbf{\#P}$
%     \item (*) Prove or disprove the following $\forall n \in \mathbb{N}_{\geq 2}$:
% \[
% \exists c \in \mathbb{N}:  \scn{\#SourceOrExcess}(n, 1) \subseteq^c \scn{PureCircuit}
% \]
% \item (*) Prove or disprove the following claim: $\forall L \in \textbf{PPAD}$
%     $$
%     \exists c \in \mathbb{N}: \#L \subseteq^c \scn{\#PureCircuit}
%     $$
% \end{enumerate}
%
Below we will be representing the development portion of the project.

\begin{enumerate}[label=S.\arabic*)]
    \item Visualise a pure circuit
    \item Generate a solution
    \item Count number of solutions for smaller scales 
\end{enumerate}


We will compile the rest of the report, based on our current findings, how we modified our objectives to
the current ones as well as general reflections.


\subsection{Core set of results}

Below we will present our main list of findings up to this point:

\begin{theorem}[ND-Brouwer to PureCircuit]
    Given $f(x) \triangleq 20$
    $$
    \scn{\#Brouwer}^{f_3} \subset^f \scn{\#PureCircuit}
    $$
\end{theorem}

\begin{corollary}
    For any $d \in \mathbb{N}_{\geq 2}$, we can define $f(\cdot) = \sum_{i = 1}^{d -1}  \binom{d}{i} 2^i$
    such that:
    $$
    \scn{\#Brouwer}^{d} \subset^f \scn{\#PureCircuit}
    $$ 
\end{corollary}



\begin{theorem}
    For any $d \in \mathbb{N}_{\geq 2}$, we can define $f(\cdot) = 5$
    such that:
    $$
    \scn{\#Brouwer}^d \subseteq^f \scn{\#PureCircuit}
    $$ 
\end{theorem}


Throughout our search, we stumbled upon
various variants of the problem as well as reductions and claims from other
subclasses of PPAD or Hazard-Free logic

\begin{proposition}
    Weaker variants of PureCircuit can result to parsimonious reductions to the EndOfLine problem
    as such:
    \begin{gather*}
        \scn{\#AcyclicBPureCircuit} \subseteq \scn{\#EndOfLine} \\
        \scn{\#PermutationFreeBPureCircuit} \subseteq \scn{\#EndOfLine}
    \end{gather*}
\end{proposition}


In addition, we were able to show that for different promise problems,
we can find reductions to the PureCircuit problem
%
\begin{proposition}
    $$
        \scn{\#PUnsatHazard} \subseteq  \scn{\#PureCircuit}
    $$
\end{proposition}

Lastly we introduce the following proposition
\begin{proposition}
    Given function $F: \mathbb{T}^n \to \mathbb{T}^n$
    where $\mathbb{T} \triangleq \{0, 1, \bot\}$ and $F$ is a \textbf{natural}
    function:
$$
\exists x^\star \in \mathbb{T}^n: F(x^\star) = x^\star
$$
\end{proposition}
%

The idea above is based on the Tarski Fixed point theorem.
Using fixed points with the Kleene logic set, has been used in the past
by Kozer as seen in his book \cite{kozen2006theory}, but we
are extending such ideas to any possible functions or gates
that use monotone gates. Doing that allows us to make the following observation.

\begin{proposition}
    $$
    \scn{\#HazardTarski} \subseteq \scn{\#PureCircuit}
    $$
\end{proposition}

