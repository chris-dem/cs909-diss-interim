Over the last couple of years, there has been a revolutionary
initiative in the field of combinatronics.
Combinatronics has been a field of study in mathematics that primarily focused on
the notion of counting objects with certain properties. Over time, this notion
has shifted, especially in the subfield of algebraic combinatorics, where
there is no clear notion of the object that we are counting,
and the numbers express something more abstract \cite{pak_WhatCombinatorialInterpretation_2022}.
This gave a need to be able to assign a combinatorial interpretation to such numbers.
Being able to find such definitions or interpretations can be very important
as it allows us to utilise tools from combinatorics and allow us to understand
and reveal hidden structures and properties \cite{pak_WhatCombinatorialInterpretation_2022}. 
Moreover, there are several problems or numbers such as \textit{Kronecker coefficients} \cite{makar_AnalysisKroneckerProduct_1949},
whose combinatorial interpretation, would give a step towards the resolution of the $\textit{P} \neq \textit{NP}$
conjecture \cite{ikenmeyer_WhatWhatNot_2022, ikenmeyer_VanishingKroneckerCoefficients_2017}.


% To formalise that idea, Pak et al. has argued that if a function $f$
% belongs to $\textit{\#P}$, it implies that $f$ has a combinatorial interpretation
% \cite{pak_WhatCombinatorialInterpretation_2022, ikenmeyer_WhatWhatNot_2022}. We invest
In our current work, we focus on extending the work done by
Ikenmeyer et al., where they focused on the creation of frameworks
that determines whether $f \in^? \textbf{\#P}$, by looking
at the subclasses of $\textsc{\#TFNP}$ problems.
In our work we focus specifically on the $\textsc{PPAD}$ class of problems, under the lens of $\text{PureCircuit}$,
which utilises sequential Kleene-based circuits, to find satisfying assignments.
We hope to uncover many insights of the $\textsc{\#PureCircuit} -1$ problem 
as well as the limits of $\textsc{\#PPAD} -1$.

% In their paper, they were able to show that for a subclass of
% problems, also known as $\textit{PPAD} \subseteq \textit{TFNP}$,
% different $\textit{PPAD-complete}$ problems, may or may not have a combinatorial
% interpretation.


\subsection{Project objectives}

Below we will present our table of objectives. We will denote updated objectives with
(*), new objectives with (!), completed objectives with (\checkmark).

\begin{enumerate}[label*=R.\arabic*)]
    \item Find a parsimonious reduction from the $\textit{EndOfLine}$ to $\textit{PureCircuit}$.
    \item Improve the combinatorial bound between the reduction from EndOfLine to PureCircuit
    \item Demonstrate that $\textbf{\#}\text{PPAD}(PureCircuit)- 1 \not\subseteq \textbf{\#P}$
    \item Prove or disprove the following
\[
\forall n \in \mathbb{N}_{\geq 2} \exists c \in \mathbb{N}:  \scn{\#SourceOrExcess}(n, 1) \subseteq^c \scn{\#PureCircuit}
\]

\end{enumerate}

Below we will be representing the development portion of the project.

\begin{enumerate}[label=S.\arabic*)]
    \item Visualise and verify a \textsc{PureCircuit} instance.
    \item Generate a solution given a \textsc{PureCircuit} instance for a small number of nodes.
    \item Count the number of solutions of a  \textsc{PureCircuit} instance for a small number of nodes.
\end{enumerate}


We will compile the rest of the report, based on our current findings, how we modified our objectives to
the current ones as well as general reflections.


\subsection{Research Question}

Our main objectives revolve around the conjecture in \ref{conj:ppad-1-hardness}, where
we investigate the boundaries of $\scn{\#PPAD}-1$ with the help of the \textsc{PureCircuit} problem.
\begin{conjecturebox}{$\scn{\#PPAD}-1$ hardness}{ppad-1-hardness}
    Every lanaguage in \scn{PPAD} can be parsimoniously reduced up to some polynomial factor, to the
    pure circuit problem,  or more formally:
    $$
    \forall L \in \scn{PPAD}, \exists f \in n^{O(1)}: 
    \#L \subseteq^f \scn{\#PureCircuit}
    $$
\end{conjecturebox}

%
% Below we will present our main list of findings up to this point:
%
% \begin{theorem}[ND-Brouwer to PureCircuit]
%     Given $f(x) \triangleq 20$
%     $$
%     \scn{\#Brouwer}^{f_3} \subset^f \scn{\#PureCircuit}
%     $$
% \end{theorem}
%
% \begin{corollary}
%     For any $d \in \mathbb{N}_{\geq 2}$, we can define $f(\cdot) = \sum_{i = 1}^{d -1}  \binom{d}{i} 2^i$
%     such that:
%     $$
%     \scn{\#Brouwer}^{d} \subset^f \scn{\#PureCircuit}
%     $$ 
% \end{corollary}
%
%
%
% \begin{theorem}
%     For any $d \in \mathbb{N}_{\geq 2}$, we can define $f(\cdot) = 5$
%     such that:
%     $$
%     \scn{\#Brouwer}^d \subseteq^f \scn{\#PureCircuit}
%     $$ 
% \end{theorem}
%
%
% Throughout our search, we stumbled upon
% various variants of the problem as well as reductions and claims from other
% subclasses of PPAD or Hazard-Free logic
%
% \begin{proposition}
%     Weaker variants of PureCircuit can result to parsimonious reductions to the EndOfLine problem
%     as such:
%     \begin{gather*}
%         \scn{\#AcyclicBPureCircuit} \subseteq \scn{\#EndOfLine} \\
%         \scn{\#PermutationFreeBPureCircuit} \subseteq \scn{\#EndOfLine}
%     \end{gather*}
% \end{proposition}
%
%
% In addition, we were able to show that for different promise problems,
% we can find reductions to the PureCircuit problem
% %
% \begin{proposition}
%     $$
%         \scn{\#PUnsatHazard} \subseteq  \scn{\#PureCircuit}
%     $$
% \end{proposition}
%
% Lastly we introduce the following proposition
% \begin{proposition}
%     Given function $F: \mathbb{T}^n \to \mathbb{T}^n$
%     where $\mathbb{T} \triangleq \{0, 1, \bot\}$ and $F$ is a \textbf{natural}
%     function:
% $$
% \exists x^\star \in \mathbb{T}^n: F(x^\star) = x^\star
% $$
% \end{proposition}
% %
%
% The idea above is based on the Tarski Fixed point theorem.
% Using fixed points with the Kleene logic set, has been used in the past
% by Kozer as seen in his book \cite{kozen2006theory}, but we
% are extending such ideas to any possible functions or gates
% that use monotone gates. Doing that allows us to make the following observation.
%
% \begin{proposition}
%     $$
%     \scn{\#HazardTarski} \subseteq \scn{\#PureCircuit}
%     $$
% \end{proposition}
%
