\subsection{Theory Crafting Next steps}

Our next steps will focus mainly on two directions. One is lowering the bound as we declare in our conjecture with the
usage of snake embeddings. We believe that each dimension can be parsimoniously reduced with another. This implies
that, if we can show that $\forall n \in \mathbb{N}_{\geq 3}: \textsc{\#nD-StrongSperner} \subseteq \textsc{\#2D-StrongSperner}$
we get to prove our conjecture.


With regards to, our second direction, we want to emphasize on the hardness of \textsc{PureCircuit}.
Or more specifically we will investigate as to if any of the following statements hold true.

\begin{gather*}
    \exists n, \alpha \in \mathbb{N}_{\geq 2}:
    \textsc{\#SourceOrExcess}(\alpha,1) - 1  \subseteq \textsc{\#nD-StrongSperner} -1 \\
    \exists n \in \mathbb{N}_{\geq 2}: \textsc{\#SourceOrExcess}(n,1) - 1 \subseteq \textsc{\#PureCircuit} -1 \\
\end{gather*}

We believe the above two should be our next steps in order to get close to the main question of our project.
Ideally generalising, the above reductions to the infinity hierarchy of
$\textsc{\#SourceOrExcess}(\cdot,1)$ would be ideal as it may lead to finding an overall upper bound
towards the entire $\textsc{\#PPAD} -1$ class.


\subsection{Software Next Steps}

In the next steps we aim to complete the tasks that were mentioned in the above table,
as well as finding a solution or even better finding all possible solutions. Of
course due to the nature of the problem, finding a single or all solutions for big instances
is computationally difficult. We aim to investigate a method introduced by Eichelberger, where
he utilised repeated applications of the circuit to detect oscillations and replace them with
$\bot$ values. The usage of that procedure was to detect hazard but aim to utilise that procedure
as a heuristical approach to reach a satisfying assignment. We do not primarily aim
to fully optimise these algorithms, as the project focuses more on analysing the theoretical findings
of \textsc{PureCircuit}.


Alternatively, one can use some facts of \textsc{PureCircuit} showed by Deligkas et al.
\cite{deligkas_PureCircuitTightInapproximability_2024}, were under specific 
gate sets, one can find a \textsc{PureCircuit} instance in polynomial time.
Additionally we were able to find instances or variants of \textsc{PureCircuit}
that are much easier. Conditions like acyclicity make our problem much easier to deal with
or permutation free circuit, make the problem easier to solve.
But just because a solution can be easy to find, does not imply that finding the total number of solutions
is easy, as \cite{valiant_ComplexityComputingPermanent_1979} indicated, or from a more direct
example by \cite{arora_ComputationalComplexityModern_2009}, where
$$
\textsc{HamiltonianCycle} \leq_P \textsc{\#Cycle}
$$





