\subsection{Reflections}

Overall we progressed slower than originally anticipated. This 
is due to the time required to familiarise ourselves with the necessary
literature.
In our original Gantt chart, we had anticipated the software portion of our project
to be nearing completion. We were not able to meet the deadline due to the complexity
of the topic. In hindsight, this allowed us to refine our project and plan ahead more effectively. 

\subsection{Lessons Learned}

As this is our first research project, we gained many valuable lessons from this experience.
We quickly realised that such projects require a much more extensive literature review than anticipated, 
and in general working on multiple problems is more beneficial than focusing on one.
Moreover, when it comes to theorising and proof development, we faced many failures.
We understood that we need to not only establish the correctness of the proof,
but also to examine its implications and assess whether they are reasonable.
However, from each failed attempt, we were able to comprehend our problem in greater depth,
refining our next steps more effectively.
Lastly, we developed a procedure to research more methodically. We often found it helpful
to utilise our canvas tools, as seen in \ref{fig:theory:obsidian-canvas}, where
a bird's-eye approach allowed us to make connections between problems more effectively.
We hope that the remaining duration of the project will reflect our aforementioned realisations
and allow us to achieve our milestones.


% Over the last couple of months I made a lot of mistakes whilst working with the project.
% Was just not focusing enough or spending enough time to do the necessary research needed
% to comprehend the topic. Moreover, I spent time focusing on the wrong aspects of my assignment.
% An example of that is referred to the sections were we showed variants of our pure circuit
% that have a combinatorial interpretation. Another core issue I faced was the lack
% of experience when dealing with more theoretical based project. This led me to
% trying to tackle the problem head on without exploring alternative avenues. For example 
% we can observe that our main reduction from the \textit{EndOfLine} came
% when using topological problems instead. 
%
% \subsection{Lessons Learned}
%
% Up to the current phase of our project, we were able to gather several important takeaways.
% Exploring the literature and trying to work the problem under different perspectives
% may lead to potential breakthroughs. The main catalyst to our core finding, came
% when looking into problems such as \textsc{EOPL} or more specifically
% the \textsc{OPDC} problem, described by Fernley et al. \cite{fearnley_UniqueEndPotential_2020}.
% \textsc{OPDC} can be described as a simpler version of the traditonal n-dimensional Brouwer problem
% where instead, we have a unqiue fixed point and our function will point towards the unique solution.
% Moreover, deeper look into Kleene-logic or more specifically \textit{Hazard-free} circuits allowed
% us to exploit the $\bot$ value to compute multiple input simultaneously \cite{ikenmeyer_ComplexityHazardfreeCircuits_2019}.

% Using the two aforementioned observations as well as looking at the original reduction again allowed us to achieve
% the desired breakthrough.
%
% In order to accomplish the above research, we have also learned how to research more effectively.
% Exploitation of LLMs such as \texttt{ChatGPT}, \texttt{Perplexity} and more, we are able to
% approach the problem from multiple perspectives. Moreover, thought organisation and cataloguing with
% \texttt{Obsidian}
% also came crucial when handling a large volume of ideas and information over a long period of time,
% as explained in the previous sections. Last but not least, utilisation of books,
% related research and textbooks also came important when it came to theory crafting and idea extraction.
%
% Last but not least, the general approach to the project changed over the course of time.
% Understanding that the journey of discovery as important as the discovery itself,
% made the project more enjoyable. Having the freedom to explore areas and expand
% the horizons of what we know, is what keeps this project exhilarating.
% Ultimately, this mindset helped the project stay engaging and allowed for continuous learning throughout.
%



