% LTeX: enabled=true


% \subsection{Important Notation}
%
% To avoid ambiguity, we introduce the following notation 
% \begin{itemize}
%     \item $\forall n \in \mathbb{N}: [n] \triangleq \{1, ..., n\}$
%     \item $\forall n \in \mathbb{N}: [n]_0 \triangleq [n] \cup \{0\}$
%     \item $\forall n \in \mathbb{N}: \bar{n}$ implies the unary representation of the number $n$ and $n$ by default implies the binary
%     \item $A^B \triangleq \{f \mid f: A \to B\}$
%     \item $\mathbb{B} \triangleq \{0,1\}$
%     \item $\mathbb{N} \triangleq \mathbb{Z}_{\geq 1}$
%     \item $\mathbb{N}_0 \triangleq \mathbb{Z}_{\geq 0}$
% \end{itemize}
%
%
\subsection{Important Notation}

To avoid ambiguity, we introduce the following notation conventions used throughout the paper.
For any $n \in \mathbb{N}$, we write $[n] \triangleq \{1, \ldots, n\}$  and $[n]_0 \triangleq [n] \cup \{0\}$. 
When using the notation $\bar{n}$, we refer to the unary representation of the number $n$; the plain symbol $n$
denotes the binary representation by default.
Given sets $A$ and $B$, we denote by $A^B \triangleq \{ f \mid f : A \to B \}$
the set of all functions from $B$ to $A$.
We define the Boolean domain as $\mathbb{B} \triangleq \{0, 1\}$,
the natural numbers as $\mathbb{N} \triangleq \mathbb{Z}_{\geq 1}$,
and the non-negative integers as $\mathbb{N}_0 \triangleq \mathbb{Z}_{\geq 0}$.


\subsection{Background overview}
As mentioned in the introduction, the current project works as an interaction
of three different fields: 1. counting complexity and combinatorial interpretation, 
2. total search problems and PPAD and 3. Kleene logic.


\subsubsection{Counting Complexity and Combinatorial Interpretations}
As we previously mentioned, we say that an object or a structure $f$
has a combinatorial interpretation if $f \in \textbf{\#P}$. $\textbf{\#P}$
is a complexity class created by Valiant \cite{valiantComplexityComputingPermanent1979},
to define a formal combinatorial framework in complexity theory.


\begin{definitionbox}{$\textsc{\#P}$ Complexity Class}{sharpp-class}
    $\textbf{\#P}$ is a class of functions $f: \{0,1\}^* \to \mathbb{N}$
    such that: there exists a polynomial time non-deterministic
    Turing machine $N: \{0,1\}^* \to \{0,1\}$ such that
    $$
    \forall w \in \{0,1\}^*: f(w) = \textit{\#acc}_N(w)
    $$
    Where $\textit{\#acc}_N(\cdot)$, denotes the number of accepting paths.
    Equivalently, we may also use the definition of: There
    exists a polynomial deterministic TM $M$, and
    $p : \mathbb{N} \to \mathbb{N}$ such that $p \in n^{O(1)}$, we have:
    $$
    f(w) = \Big|\Big\{v \in \{0,1\}^{p(|w|)} \mid M(w, v) =1 \Big\}\Big|
    $$
\end{definitionbox}

$\textbf{\#P}$ was initially created by \cite{valiantComplexityComputingPermanent1979},
to demonstrate, that even if we have a problem $L \in P$, $\#L$ can be computationally
hard to compute, by providing an example of computing the permanent of
a 01-matrix, with number of perfect matchings. 

As we can see, $\textbf{\#P}$ allows us to define a set of objects,
whose cardinality equals $f(w)$. Core reasoning for choosing
$\textbf{\#P}$ to define combinatorial objects is mainly for the 
following two reasons \cite{ikenmeyerPositivitySymmetricGroup2024}: 

\begin{enumerate}
    \item By polynomially bounding words, we avoid cases such as: $f(w) = \{1, \hdots, f(w)\}$
    \item Current framework allows us to work with $f(\cdot)$, whose direct computation can be computationally hard
\end{enumerate}


The current framework was used in several papers such as
\cite{ikenmeyerWhatWhatNot2022} and \cite{ikenmeyerPositivitySymmetricGroup2024}
where they were able to use tools from complexity theory to show that
many structures do or do not have a combinatorial interpretation.
For the purposes of the current project, we are focusing on \cite{ikenmeyerWhatWhatNot2022},
where they demonstrated how several $\textsc{TFNP}$ problems, 
change in complexity as we ignore one of its solutions.


Any class of problems in complexity theory utilises the notion of reduction
to indicate the similarity between problems. For counting complexity
this is referred to as \textit{parsimonious reductions}.


\begin{definition}[Parsimonious reductions]
    Let $R, R'$ be search problems and let $M$ be a Karp reduction of
    $S_R = \{x \mid R(x) \neq \emptyset \}$ to $S_{R'} = \{x \mid R'(x) \neq \emptyset \}$.
    We say $f$ is \textbf{parsimonious} if:
    $$
    \forall x \in S_R : |R(x)| = |R'(f(x))|
    $$ 
\end{definition}

\subsubsection{Total Search Problems and PPAD}
When talking about search problems, we are using the following definition:

\begin{definition}[Search Problems and Total Search Problems]
    \textbf{Search problems} can be defined as relations $R \subseteq \{0,1\}^* \times \{0,1\}^*$,
    where given $x \in \{0,1\}^*$, we want to find $y \in \{0,1\}^*$  such that $x Ry$.

    \textbf{Total Search problems} are search problems such that:
    $$
    \forall x \in \{0,1\}^*, \exists y \in \{0,1\}^* : xRy
    $$
\end{definition}

Using the above, we can define the following complexity classes

\begin{definition}[FP, FNP, TFNP]
    \textbf{FP} are \textit{search problems} such that there exists poly-time TM $M$
    such that $M(x) = y$ where $x Ry$.\\
    \textbf{FNP} are \textit{search problems} such that there exists poly-time TM $M: \{0,1\}^* \to \{0,1\}$
    and a poly function $p : \mathbb{N} \to \mathbb{N}$ such that:
    $$
    \forall x \in \{0,1\}^*, y \in \{0,1\}^{p(|x|)}: xRy \iff M(x,y) = 1
    $$
    Lastly $\textbf{TFNP} = \{L \in \textbf{FNP} \mid L \text{ is total}\}$
\end{definition}

Our current work focuses on a specific subclass of $\textbf{TFNP}$ problems
which is defined as follows:

\begin{definition}[\textit{EndOfLine} problem]
    Given circuits $S, P \in \{0,1\}^n \to \{0,1\}^n$ such that $S,P \in n^{O(1)}$
    we define a directed graph $G = (V,E)$, such that $V= \{0,1\}^n$ and $E$ defined as:
    $$
    E = \{(x,y) \in V^2: S(x) = y \wedge P(y) = x\}
    $$
    We define are source or sinks $\forall v \in V: \textit{deg}(v) = (0,1)$ or
    $\textit{deg}(v) = (1,0)$, respectively. We want to output
    $v$ that is either a source or a sink.
\end{definition}


\begin{definition}[Levin Reductions]
    Given a pair of search problems $R_A, R_B$, a pair of
    computable time functions $(f,g)$ is called a Levin reduction from $R_A \to R_B$
    \begin{gather*}
        S_R \triangleq \{x \mid \exists y : xRy  \}\\
        R(x) \triangleq \{y \mid x Ry \} \\
        \forall x \in S_{R_A}, y_b \in R(f(x)):  (x , g(x, y_b)) \in R_A
    \end{gather*}
\end{definition}

\begin{definition}[\textbf{PPAD} complexity class]
    \textbf{PPAD} is defined as the set of search problems that
    are levin reducible to the \scn{EndOfLine} problem
\end{definition}


\textbf{PPAD} has been created by Papadimitriou \cite{papadimitriouComplexityParityArgument1994}
to demonstrate a subset of problems in \textbf{NP} that are guaranteed to have
a solution but can be very difficult to find. We will define
several problems of interest as they will be referenced in later sections.

\begin{definition}[\textit{SourceOrExcess} problem]
    We define as $\textit{SourceOrExcess}(k,1)$ for $k \in \mathbb{N}_{\geq 2}$
    the search problem as such: Given a poly-sized successor circuit $S : \{0,1\}^n$
    and a set of predecssor poly-sized circuits $\{P_i\}_{i \in [k]}$, we define
    the graph $G = (V,E)$ such that, $V = \{0,1\}^n$ and $E$ as:
    $$
    \forall x, y \in V^2: (x,y) \in E \iff (S(x) = y) \wedge \bigvee_{i \in [k]} P_i(y) = x
    $$
    We ensure that $0^n$ is as sink, meaning $\text{deg}(0^n) = (0,1)$.
    A valid solution is a vertex $v$ such that $\textit{in-deg}(v) \neq \textit{out-deg}(v)$
\end{definition}



\begin{definition}[StrongSperner problem]
    \textbf{Input}: A boolean circuit computing a labelling $\lambda: [M]^N \to \{-1, 1\}^N$
    satisfying the following boundary conditions $\forall i \in [N]$:
    \begin{itemize}
        \item if $x_i = 1 \implies [\lambda(x)]_i = +1$
        \item if $x_i = M \implies [\lambda(x)]_i = -1$
    \end{itemize}
    \textbf{Output}: A set of points $\{x^{(i)}\}_{i \in [N]} \subseteq [M]^{[N]}$, such that:
    \begin{itemize}
        \item \textit{Closessness condition}: $\forall i,j \in [N]: \|x^{(i)} - x^{(j)}\|_{\infty}$
        \item \textit{Covers all labels}:
            $$
            \forall i \in [N], \ell \in \{-1, +1\}, \exists j \in [N]: [\lambda(x^{(j)})]_{i} = \ell
            $$
    \end{itemize}
\end{definition}


The above is a generalised variant of the tradtional Sperner problem to
a grid of dimensions $N$ and width of $M$. 
Throughout literature the same variants of the problem or specifications
have been defined using sperner or discrete brouwer. For the sake 
of clarity, we will stick to $\textsc{StrongSperner}$ and
we will refer to $\textsc{StrongSperner}^n$ as the problem
with a fixed dimensionality of $n$. This problem
is crucial as it show $\textbf{PPAD}$-Hardness as well as
relating to the reduction of various problems.
We mainly use this to show counting arguments with respect to the
\textsc{EndOfLine} as people have indicated a parsimonious reduction
between $\textsc{StrongSperner}^2$ to a variant of the EndOfLine
and $\textsc{StrongSperner}^3$ to the EndOfLine problem but with
different colourings.

%% Add citations


These problems have found
connections with various other problems such as Nash Equilbria,
Fixed point theorems or Sperner Lemmas. The current project looks
on the counting complexity of such problems as despite two problems
being PPAD-complete, Ikenmeyer et al. has showed several examples where
their counting difficulty can differ vastly. Examples of such statements
can be summarised as:

\begin{gather*}
    \textbf{\#PPAD}(\textsc{SourceOrSink})-1/2  = \textbf{\#P}\\
    \textbf{\#P}^A \not\subseteq \textbf{\#PPAD}(\textsc{SourceOrExcess(2,1)})^A-1
\end{gather*} 


Currently we are mainly focused into the study of a certain 
$\textbf{PPAD}\textit{-complete}$ problem, known as $\textsc{PureCircuit}$

\subsubsection{Kleene Logic and Hazard-Free Circuits}

Kleene logic, is the type of system that extends the traditional %% Please triple check it just in case you get nuked for this
binary system to: $\mathbb{T} = \{0, \bot,1 \}$. 
Study of such systems has an impact to both
the development of robust systems and from a theoritical point of view,
people developed systems such as \textit{Alternative Turing mahcines}, as well as
showing important correlations between the complexity of monotone circuits
and the robustness of hazard free circuits. 

We want to introduce the following concepts:

\begin{definition}[Kleene Resolutions]
    Given $x \in \mathbb{T}^n$, we define the \textit{resolutions} of $x$, using the following notation:
    $$
    \scn{Res}(x) \triangleq \big\{ y \in \mathbb{B}^n \mid \forall j \in [n]: x_i \in \mathbb{B} \implies x_i = y_i \big\}
    $$
\end{definition}

\begin{definition}[Hazard circuit]
    A \textit{circuit} $C$, on $n$ inputs has \textbf{hazard} at $x \in \mathbb{T}^n \iff C(x) = \bot$
    and $\exists b \in \mathbb{B}, \forall r \in \scn{Res}(x): C(r) = b$
\end{definition}


\begin{definition}[Natural function]
    A function $F: \mathbb{T}^n \to \mathbb{T}$ can be computed by stable values and is monotone with respect to $\preceq$
\end{definition}

\begin{definition}[K-bit Hazard]
    For $k \in \mathbb{N}$ at $x \in \mathbb{T}^n \iff C$  has a \textit{hazard} at $x$ and $\bot$ appears at most $k$ times in $x$
\end{definition}



